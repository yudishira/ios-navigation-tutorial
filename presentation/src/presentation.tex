\documentclass[serif,mathserif]{beamer}


\usepackage{fleqn}
%%\usepackage{modefs}
%% \usepackage{math}
\usepackage[utf8]{inputenc}
\usepackage{amsmath, amsfonts, epsfig, xspace}
\usepackage{pstricks,pst-node}
\usepackage{multimedia}
\usepackage[normal,tight,center]{subfigure}
\usepackage{listings}
\setlength{\subfigcapskip}{-.5em}
\usepackage{beamerthemesplit}
\renewcommand\sfdefault{phv}
\renewcommand\familydefault{\sfdefault}
\usetheme{default}
\usepackage{color}
\useoutertheme{default}
\usepackage{texnansi}
\usepackage{marvosym}
\definecolor{bottomcolour}{rgb}{0.32,0.3,0.38}
\definecolor{middlecolour}{rgb}{0.08,0.08,0.16}
\setbeamerfont{title}{size=\Huge}
\setbeamercolor{structure}{fg=gray}
\setbeamertemplate{frametitle}[default]%[center]
\setbeamercolor{normal text}{bg=black, fg=white}
\setbeamertemplate{background canvas}[vertical shading]
[bottom=bottomcolour, middle=middlecolour, top=black]
\setbeamertemplate{items}[circle]
\setbeamerfont{frametitle}{size=\huge}
\setbeamertemplate{navigation symbols}{} %no nav symbols

\author[Vinicius Miana]{Vinicius Miana}

\title[Short Title\hspace{2em}\insertframenumber/\inserttotalframenumber]{Tutorial Navigation}

\date{12 de fevereiro de 2014} %leave out for today's date to be insterted

\institute{Universidade Presbiteriana Mackenzie}

\begin{document}


\lstset{language=[Objective]C,
  keywordstyle=\color{blue},
  keywords=[2]{UIApplication,NSDictionary,UIViewController,UINavigationController,UITableViewCell},
  keywordstyle=[2]\color{orange},
  stringstyle=\color{red},
  commentstyle=\color{green},
  breaklines=true,
  basicstyle=\footnotesize
}

\maketitle

\section{UINavigationController}  


\begin{frame}
  \frametitle{Desafio: Como deixar as navegações no seu aplicativo fácil e intuitiva.}
   Questões:
  \begin{itemize}
  \item Como funciona o UINavigationController? 
  \item Quais são os componentes envolvidos no seu funcionamento?
  \item Quais são os métodos do UIViewController que lidam com a navegação?
  \item ?
  \end{itemize}
\end{frame}


\begin{frame}
  \frametitle{UINavigationController}
  \begin{itemize}
  \item É um controller especial que mantém uma pilha com ViewControllers e suas Views correspondentes.
  \item Ele precisa ser inicializado com um controlador raiz.
  \item Para adicionar e remover controllers no stack usamos os métodos pushViewController e popViewController
  \end{itemize}
\end{frame}



\begin{frame}
  \frametitle{Para iniciar o projeto com um Navigation Controller}
  \begin{itemize}
  \item Crie um projeto vazio
  \item Adicione o NavigationController como propriedade do AppDelegate
  \item Instancie-o e defina-o como root viewController no método didFinishLaunchingWithOptions
  \end{itemize}
\end{frame}

\begin{frame}[fragile]
  \frametitle{Iniciando a app com Navigation Controller}
  \begin{lstlisting}
    - (BOOL)application:(UIApplication *)application didFinishLaunchingWithOptions:(NSDictionary *)launchOptions {
      UIViewController *viewController = [[UIViewController alloc] initWithNibName:nil bundle:nil];
      viewController.title = @"Primeira View";    
      self.navigationController = [[UINavigationController alloc]
        initWithRootViewController:viewController];
      self.window = [[UIWindow alloc]
        initWithFrame:[[UIScreen mainScreen] bounds]];
      self.window.rootViewController = self.navigationController;

      self.window.backgroundColor = [UIColor whiteColor];
      [self.window makeKeyAndVisible];
      return YES;
    }
  \end{lstlisting}
\end{frame}

\begin{frame}[fragile]
  \frametitle{Navegando}
  A view instanciada desta maneira apresenta uma barra de navegação na parte superior da
  tela. Para mostrar novas telas devemos usar:
  \begin{lstlisting}
    [self.navigationController pushViewController:umNovoViewController
      animated:YES];  
  \end{lstlisting}
  Ao apresentar a nova View gerenciada por umNovoViewController, a barra de navegação apresenta também um botão de back que permite voltar para a view inicial.
\end{frame}  

  \begin{frame}[fragile]
    \frametitle{Navegando}
  A pilha de ViewControllers mostrados é mantida pelo NavigationController em um array que pode ser manipulado diretamente. Através da propriedade viewControllers:
  \begin{lstlisting}
    NSArray *currentControllers = self.navigationController.viewControllers;
  \end{lstlisting}
  Ao manipular diretamente este array perdemos a propriedade de animação. Para mantê-la, devemos
  copiar o array, alterar seu conteúdo e depois alterar o stack do navigationController mantendo a animação na transição com o método:
  \begin{lstlisting}
  [self.navigationController setViewControllers:newControllers
    animated:YES];
  \end{lstlisting}  
\end{frame}


\begin{frame}[fragile]
  \frametitle{Adicionando botões na barra de navegação}
  O UIViewController tem uma propriedade navigationItem que permite que um view controller interaja com
  a barra de navegação. Olhe na documentação a classe UINavigationItem.
  Para colocar um botão em uma view, devemos:
  \begin{itemize}
    \item Crie uma instância de UIBarButtonItem e atribua à propriedade rightBarButtonItem
      \begin{lstlisting}
        self.navigationItem.rightBarButtonItem =
        [[UIBarButtonItem alloc] initWithTitle:@"Um Botao"
          style:UIBarButtonItemStylePlain target:self action:@selector(umaAcao:)];
        \end{lstlisting}
      \item Onde devemos colocar este código?
  \end{itemize} 
\end{frame}


\begin{frame}
  \frametitle{O desafio - Criar um dicionário infantil}
  \begin{itemize}
  \item Fork e Clone o projeto do github
  \item BRONZE - Faça com que cada view mostre uma letra, uma palavra começada com a letra e seja capaz de ler a palavra em voz alta.
  \item PRATA - O código não está bem feito. Precisamos de 26 view controllers?
  \item OURO - Dá para evitar manter um stack de 26 views ao chegarmos na letra Z?
  \end{itemize}
\end{frame}



  


%   \begin{figure}[t]
% \centering
%\subfigure[Exemplos de estilo de células]{
  %%      \includegraphics[width=5cm]{img/tableCellStyles.png}
%}
%\end{figure}




\begin{frame}
  \frametitle{Questions}
\end{frame}
\end{document}
